

\section{Experimental setup}
\label{sec:experiment}

The analysis is performed with data samples in pp collisions at $\sqrt{s} = 13$~TeV collected from 2016 to 2018 and in p--Pb collisions at $\sqrt{s_\mathrm{NN}} = 5.02$~TeV in 2016 during the LHC Run 2 period. A full description of the ALICE detector and performance in the LHC Run 2 can be found in Refs~\cite{Aamodt:2008zz,Abelev:2014ffa}. The analysis utilizes the V0 detector~\cite{Abbas:2013taa}, the Inner Tracking System (ITS)~\cite{aliceITS}, and the Time Projection Chamber (TPC)~\cite{aliceTPC}. 

The V0 detector consists of two stations on both sides of the interaction point, V0A and V0C, each made of 32 plastic scintillator tiles, covering the full azimuthal angle within the pseudorapidity intervals $2.8 < \eta < 5.1$ and $-3.7 < \eta < -1.7$, respectively. The V0 provides a minimum-bias (MB) trigger in both pp and p--Pb collisions and an additional high-multiplicity (HM) trigger in pp collisions. The minimum bias trigger is obtained by a time coincidence of V0A and V0C signals. The charged particle multiplicity is determined based on the sum of the V0A and V0C signals, which is denoted as V0M. The high-multiplicity trigger requires that the V0M signal exceeds 5 times the mean value measured in MB collisions, selecting the 0.1\% of MB events that have the largest V0M multiplicity. The analyzed data samples of MB and high-multiplicity pp events at $\sqrt{s}=$~13 TeV correspond to integrated luminosities ($\mathcal{L}_\mathrm{int}$) of 19 nb$^{-1}$ and 11 pb$^{-1}$, respectively~\cite{ALICE-PUBLIC-2016-002}. In p--Pb collisions at $\sqrt{s_\mathrm{NN}} = 5.02$ TeV, the number of events corresponding to $\mathcal{L}_\mathrm{int} = 3$ nb$^{-1}$ is used for the analysis. 

The primary vertex positions are reconstructed from signals measured by the Silicon Pixel Detector (SPD)~\cite{Santoro2009:ALICESPD}, consisting of the two innermost layers of the ITS. The reconstructed primary vertices are required to be within 8 cm of the nominal interaction point along the beam direction. The probability of pileup events is estimated to be about 0.6\% for MB and high multiplicity events in pp collisions. Pileup events are determined and rejected if the longitudinal displacement of the secondary vertex is greater than 0.8 cm in pp collisions. The pileup probability is estimated to be negligible in p--Pb collisions. 

Charged-particle tracks are reconstructed using the combined information of the ITS and TPC in a uniform magnetic field of 0.5 T along the beam direction by the solenoid. The ITS is a silicon tracker with six layers of silicon sensors, the SPD consists of the two innermost layers, the next two layers are the SDD (Silicon Drift Detector) and the outermost layers are the SSD (Silicon Strip Detector). The ITS and TPC are covering the entire azimuth range up to $|\eta|<1.4$ and 0.9, respectively, for the detection of charged particles emitted within 8 cm from the nominal vertex position $z_\mathrm{vtx}=0$ along the beam direction. Charged particle tracking is performed with the ITS and TPC capable of reconstructing tracks down to transverse momentum ($\pt$) of 0.15 GeV/$c$ with an efficiency of about 65\% and the efficiency reaches 80\% for intermediate $\pt$, 1--5~GeV/$c$. The $\pt$ resolution is about 1\% (1\%) for primary charged particles with $\pt<$~1~GeV/$c$, linearly increasing to 6\% (10\%) at $\pt\sim$ 50~GeV/$c$ in pp (p--Pb) collisions~\cite{ALICE:2018vuu}. 

The charged particle selection criteria are optimized to ensure a uniform efficiency over the sensitive TPC volume to mitigate the effects of small areas where some ITS layers are inactive in both collision systems. The selection consists of two classes of tracks. Those in the first class must have at least one hit in SPD. The tracks of the second class do not have any hits related to the SPD, and their origins are rather constrained to the primary vertex~\cite{ALICE:2012eyl}. 

\section{Analysis procedure}
\label{sec:ana}
\subsection{Two-particle angular correlations}
Two-particle angular correlations are measured as functions of the relative azimuthal angle ($\Delta\varphi$) and the relative pseudorapidity ($\Delta\eta$) between a trigger and associated particles
\begin{eqnarray}
\frac{1}{N_{\rm{trig}}} \frac{ \rm{d}\it{}^{2} N_{\rm{pair}} }{ \rm{d} \Delta\eta \rm{d}\Delta\varphi} = B(0, 0)\frac{S(\Delta\eta, \Delta\varphi)}{B(\Delta\eta, \Delta\varphi)}  \Big\lvert_{\pttrig,\,\ptassoc}\quad , 
\label{eq:corrfunction}
\end{eqnarray}
where the transverse momentum range for associated particles ($p_\mathrm{T,assoc}$) is $1<p_\mathrm{T,assoc}<4$~GeV/$c$ for different transverse momentum ranges of trigger particles ($p_\mathrm{T,trig}$).
The lower limit of $p_\mathrm{T,assoc}$ is chosen in order to avoid jet-like contributions. 
This is investigated in the large $\Delta\eta$ range that is used in this analysis ($1.6<|\Delta\eta|<1.8$), and the result is that when using a lower limit than 1 GeV/$c$, the jet-like contributions extend into the large $\Delta\eta$ range.

%where the trigger and associated particles are defined for different transverse momentum ranges, $1<p_\mathrm{T,trig}<2$ GeV/$c$ and $1<p_\mathrm{T,assoc}<4$ GeV/$c$.
The $N_\mathrm{trig}$ and $N_\mathrm{pair}$ are the numbers of trigger particles and trigger-associated particle pairs, respectively. $S(\Delta\eta, \Delta\varphi)$ corresponds to the average number of pairs in the same event and $B(\Delta\eta, \Delta\varphi)$ to the number of pairs in mixed events. 
$B (0,0)$ represents the normalization of $B(\Delta\eta, \Delta\varphi)$, and by dividing $S(\Delta\eta, \Delta\varphi)$ with $B(\Delta\eta, \Delta\varphi)/B (0,0)$ the acceptance effects are corrected for. The track reconstruction efficiency is corrected for on the right-hand side of Eq.~\ref{eq:corrfunction} as functions of $p_\mathrm{T}$ and $\eta$. 
In the previous study~\cite{ALICE:2021nir}, the tracking efficiency was calculated with a detector simulation with the PYTHIA 8 event generator and the GEANT3 transport code~\cite{Brun:1994aa}. The tracking efficiency is determined by re-weighting the primary particle composition based on a data driven method~\cite{ALICE:2018hza,ALICE:2018vuu}. Therefore, this method improves the jet-yield extraction and has no impact on the flow extraction.
For each multiplicity percentile, the pairs in mixed events are required to have primary vertices within the same 2 cm wide $z_{vtx}$ interval and the correlation functions are averaged over the vertex bins which results in the final per-trigger yield~\cite{KOPYLOV1974472:evtmixing,Adam:2016tsv}.

\begin{figure}[h!]
		\includegraphics[width=0.5 \textwidth]{figures/CorrForAN_C_0_0_0_11} 
		\includegraphics[width=0.5 \textwidth]{figures/CorrForAN_C_0_0_4_11} 
  		\includegraphics[width=0.5 \textwidth]{figures/corr_1_0_2_pPb}
		\includegraphics[width=0.5 \textwidth]{figures/corr_1_6_2_pPb}
\caption{Two-particle correlation functions as functions of $\Delta\eta$ and $\Delta\varphi$ for HM(0--0.1\%, left) and LM(60--100\%, right) events in $\sqrt{s}=13$ TeV pp (top) and $\sqrt{s_{\mathrm{NN}}}=5.02$ TeV p--Pb (bottom) collisions. The $z$-axis. The intervals of $\pttrig$ and $\ptassoc$ are 1~$<\it{p}_{\rm{T}}<$~2~GeV/$c$ in all cases.}
\label{fig:doubleridge}
\end{figure}

Two-dimensional correlation distributions in pp collisions at $\sqrt{s}=13$ TeV are shown in Fig.~\ref{fig:doubleridge} for the high-multiplicity (0--0.1\%, left) and low-multiplicity (60--90\%, right). 
The $z$-axis for the correlation yield is properly scaled in order to zoom in the larger $\Delta\eta$ region. As a result, the jet peaks are sheared off in both figures. The flow modulation structure is clearly observed in the HM class while it is not seen in the LM-template. The away-side regions are populated mostly by back-to-back jet correlations but they are reduced and compatible to the one in near side in $\Delta\eta > 1.6$.

The per-trigger yield is extracted for various multiplicity percentiles and $\pt$ intervals at large $\Delta\eta$, at $1.6<|\Delta\eta|<1.8$ to remove the near-side non-flow contributions. The per-trigger yield as a function of $\Delta\varphi$ is expressed as
\begin{eqnarray}
Y(\Delta\varphi) = \frac{1}{N_{\rm{trig}}} \frac{ \rm{d}\it{}N_{\rm{pair}} }{ \rm{d}\Delta\varphi } = \int_{1.6<|\Delta \eta|<1.8} \left( \frac{1}{\it{N}_{\rm{trig}}} \frac{ \rm{d}\it{}^{2} N_{\rm{pair}} }{ \rm{d}\Delta\eta \rm{d}\Delta\varphi} \right) \dfrac{1}{\delta_{\Delta\eta}} \rm{d}\Delta \eta \quad ,
\label{eq:pertrigger}
\end{eqnarray}
where $\delta_{\Delta\eta}=$~0.4 is the normalization factor for the whole integrated ranges to scale the per-trigger yield per unit of pseudorapidity. 
%The Zero-Yield-At-Minimum (ZYAM) procedure~\cite{Ajitanand:2005jj} is the method used to subtract the baseline of the correlation. 


\subsection{Extraction of flow coefficients from the low-multiplicity template fit method}

It is difficult to extract the flow coefficients in small systems because of the remaining strong jet fragmentation bias in the away-side region ($\Delta\varphi \sim \pi$) in Eq.~\ref{eq:pertrigger}. As discussed in Refs.\cite{ATLAS:2015hzw,ATLAS:2016yzd}, the correlation function in a high multiplicity percentile can be expressed as 
\begin{eqnarray}
Y_{\rm{HM}}(\Delta\varphi) = G~(1 + 2v_{2,2}\cos(2\Delta\varphi) + 2v_{3,3}\cos(3\Delta\varphi)) + F~Y_{\rm{LM}}(\Delta\varphi) \quad,
\end{eqnarray}
where $Y_{\rm{LM}}(\Delta\varphi)$ is the low-multiplicity template, G is the normalization factor for the Fourier component up to the third harmonic, and the scale factor $F$ corresponds to the relative away-side jet-like contribution with respect to the LM (the 60--100\%) template. This method requires $Y_{\rm{LM}}$ not to contain a peak in the near side and assumes that the away-side region originates from jet fragmentations and that the jets are not modified in HM events. The fit determines the scale factor $F$, pedestal $G$, and $v_{n,n}$. This method does not rely on the zero yield at minimum (ZYAM) hypothesis~\cite{Ajitanand:2005jj} to subtract an assumed flat combinatorial component from the LM template as done previously in Refs.~\cite{ATLAS:2012cix,ATLAS:2014qaj}. The assumption that the shape of the away-side jet shape in HM events is not modified compared to the LM template is further tested by the ATLAS Collaboration in Ref.~\cite{ATLAS:2018ngv} and its effect was found to be relatively small. In case of our kinematic ranges, since any hints of near-side peak is not found in our LM-template, any correlated modulation in away-side should be negligible. However, this method can be used to extract flow modulations for EPOS LHC and PYTHIA8 String Shoving models which have relatively large ridge signals in LM events. This is further discussed in Sec.~\ref{sec:results}.

\begin{figure}[h!]
	\centering
	\includegraphics[width=0.6 \textwidth]{figures/Fig1_FlowExt.pdf} 
	\caption{The template fit results with the biased LM-templates. The black markers shows the signal for the $0-0.1\%$ multiplicity percentile together with its fit shown as a blue band. The red squares correspond to the LM signal. The orange and green curves correspond to the extracted $v_2$ and $v_3$ signals, respectively. The $\chi^{2}$ divided by the number of degree of freedom is 0.894}
	\label{fig:flowext}
\end{figure}

Finally, $v_{n}$ are extracted, based on the observed factorization of $v_{n,n}$ to single harmonics~\cite{ATLAS:2015hzw,ATLAS:2016yzd}, using the following equation\quad,
\begin{eqnarray}
v_{n}(p_{\rm{T,trig}}) = v_{n,n}(p_{\rm{T,trig}}, p_{\rm{T,assoc}}) / \sqrt{ v_{n,n}(p_{\rm{T,assoc}},p_{\rm{T,assoc}})}
\end{eqnarray}
, where $v_{n,n}(p_{\rm{T,assoc}}$ and $p_{\rm{T,assoc}})$ are measured in 1~$<p_{\rm{T,trig}}<$~4~GeV/$c$ interval as the $p_{\rm{T,assoc}}$ range is fixed with 1~$<p_{\rm{T,assoc}}<$~4~GeV/$c$.

Figure~\ref{fig:flowext} shows the LM-template fit results for 0--0.1\% multiplicity percentile pp collisions at $\sqrt{s}$ = 13 TeV. The LM yield,  $Y_{\rm{LM}}(\Delta\varphi)$, is shown in red squares, and the extracted $v_{2}$ and $v_{3}$ as orange and green respectively. The resulting scale factor $F$ in different multiplicity percentiles and systems are summarized in Tabs.~\ref{tab:Fpp} and \ref{tab:Fpb}. They are larger for pp higher multiplicity percentiles and closer to unity for p--Pb lower multiplicity percentiles. In the previous results in Refs.~\cite{ALICE:2012eyl,ALICE:2013snk}, $F$ was assumed to be 1. The total fit is shown as a blue band and the signal-to-fit ratio is shown in the bottom panel. 
%let's add chiq value of the fit
\begin{table}[h!]
\caption{The scale factor $F$ for various multiplicity percentiles in pp collisions.}
\centering
\begin{tabular}{|c|cccc|c}
\hline
 V0M& 0--0.1\% & 1--5\% & 5--20\% & 20--60\% \\ 
 \hline
 $F$ & 1.504$\pm$0.017 & 1.414$\pm$0.030 & 1.360$\pm$0.019 & 1.208$\pm$0.015 \\  
 \hline
 \end{tabular}
 \label{tab:Fpp}
 
\end{table}

\begin{table}[h!]
\caption{The scale factor $F$ for various multiplicity percentiles in p--Pb collisions.}
\centering
\begin{tabular}{|c|cccccc|c}
 \hline
 V0A& 0--5\% & 5--10\% & 10--20\% & 0--20\% & 20--40\% & 40--60\% \\ 
 \hline
 $F$& 1.135$\pm$0.026 & 1.140$\pm$0.026 & 1.152$\pm$0.021 & 1.145$\pm$0.017 &1.092$\pm$0.015 & 1.083$\pm$0.015 \\  
 \hline
\end{tabular}
\label{tab:Fpb}
\end{table}

\begin{figure}[h!]
	\centering
	\includegraphics[width=0.6 \textwidth]{figures/Fig5_Plot_v2Mult.pdf} 
	\caption{The $Y^{\rm{frag}}$ for the near- and away-side as a function of multiplicity percentiles with both ALICE and PYTHIA data. The ratio includes the combined statistical and systematic errors in quadrature.}
	\label{fig:Ymult}
\end{figure}

The near-side jet-like yields are extracted from the near-side $\Delta\eta$ correlations, defined in $|\Delta\varphi|<$~1.3 projected on the $\Delta\eta$ axis as the following
\begin{eqnarray}
Y^{near}_{\rm{frag}} = \int_{|\Delta \eta|<1.3} \left( \frac{1}{\it{N}_{\rm{trig}}} \frac{ \rm{d}\it{}N_{\rm{pair}} }{ \rm{d}\Delta\eta } \right) \rm{d} \Delta\eta \quad.
\label{eq:Ynear}
\end{eqnarray}
The value 1.3 is chosen for the projection range in order to fully cover $\Delta\varphi$ in Eq.~\ref{eq:pertrigger}. The correlations in the same jet mainly contribute to $(\Delta\eta, \Delta\varphi) \sim (0,0)$ due to the similar out-going direction of particles inside the jet along the jet-axis. The jet fragmentation yield is defined by integrating the $\Delta\eta$ correlation over $|\Delta\eta|<1.3$ after applying the ZYAM procedure~\cite{Ajitanand:2005jj}. The ZYAM procedure is applied by finding the minimum value of $\Delta\eta$ correlations within $|\Delta\eta|<1.3$, which results in pointing the minimum position to be $|\Delta\varphi|=1.3$.

The away-side jet-like yield in data is measured from the LM-template fit method as $Y^{\rm{away, HM}}_{\mathrm{frag}} = Y^{\rm{away, LM}}_{\mathrm{frag}} \times F$, where $F$ is again the relative contribution of the away-side jet fragmentation yield between low- and high-multiplicity events. The $Y^{\rm{away, LM}}_{\mathrm{frag}}$ is then directly measured by integrating the away-side LM $\Delta\varphi$ correlation function. Since this method is not necessary for PYTHIA, where there is no flow contributions, $Y^{\rm{away}}$ is measured directly from the $\Delta\varphi$ correlation functions.

Figure~\ref{fig:Ymult} presents the $Y^{\mathrm{near}}_{\rm{frag}}$ and $Y^{\mathrm{away}}_{\rm{frag}}$, for both ALICE data and PYTHIA 8 Tune 4C as a function of multiplicity percentile in pp collisions at $\sqrt{s}=13$ TeV. The transverse momentum range for trigger particles is $1<p_\mathrm{T,trig}<2$ GeV/c and for associated particles $1<p_\mathrm{T,assoc}<4$ GeV/c. ALICE data points are shown as circle and square markers, whereas PYTHIA is shown as lines. The near-side yields are shown in black and the away-side yields in blue.
The near- to away-side ratio for ALICE and PYTHIA data is shown in the bottom panel. While PYTHIA overestimates ALICE data for both $Y^{\mathrm{near}}_{\rm{frag}}$ and $Y^{\mathrm{away}}_{\rm{frag}}$, the ratio in PYTHIA is consistent with the ALICE data in the multiplicity percentiles studied. This ratio can be explained by the pair acceptance effect caused by the limited ALICE $\eta$ acceptance~\cite{PHENIX:2006gto}, which implies that the enhanced jet yields in away-side in HM events with respect to LM events are well quantified by the LM-template method.

\begin{figure}[h!]
		\includegraphics[width=0.5 \textwidth]{figures/CorrForAN_C_SUB_0_0_0_11.jpg}
  		\includegraphics[width=0.5 \textwidth]{figures/corr_sub_fit_1_0_2_pPb}
\caption{The subtracted one as (0--0.1)\%-$F$(60--100\%) is shown on the right. The subtracted one as (0--5)\%-$F$(60--100\%) is shown on the right. Note that the near-side jet peaks exceed the chosen range of the $z$-axis. The intervals of $\pttrig$ and $\ptassoc$ are 1~$<\it{p}_{\rm{T}}<$~2~GeV/$c$ in all cases.}
\label{fig:doubleridge}
\end{figure}

The subtracted one as (0--5)\%-$F$(60--100\%) on the right. Similarly, the ones in p--Pb collisions at $\sqrt{s_{\mathrm{NN}}}=5.02$ TeV are shown in Fig.~\ref{fig:doubleridgepPb}. The $F$ values can be found in Tables~\ref{tab:Fpp} and \ref{tab:Fpb}, which come from the LM-template fit in a given multiplicity percentile as described in Sec.~\ref{sec:ana}. 


\subsection{Event-scale dependence}
As a continuation of the extraction of the flow coefficients with the LM-template fit method, different event scale selections are investigated. By selecting events that includes a hard jet or a high-$\pt$ leading particle within the mid-rapidity region, it is possible to study the impact parameter dependence of flow coefficients in pp collisions~\cite{Sjostrand:1986ep,Frankfurt:2010ea}. This event scale is set by requiring a minimum $\pt$ of the leading track ($\ptlead$) or the reconstructed jet ($\ptjet$) at midrapidity. The leading particle track requires to be within $|\eta|<0.9$ and $0<\phi<2\pi$, and the jets are reconstructed with the anti-$k_{\rm{T}}$ algorithm~\cite{Cacciari:2008gp,Cacciari:2011ma}, with $R=0.4$ for only charged particles. In this analysis the $\pt$ scheme is used as the recombination scheme. As the leading particle tracks, the jets are selected in the full azimuthal angle ($0<\phi<2\pi$) but in an $\eta$-range of $|\eta_\mathrm{jet}|<0.4$. The $\pt$ of jets $\ptjet$ is corrected for the underlying event density that is measured using the $k_{\rm{T}}$ algorithm with $R=$~0.2~\cite{Acharya:2018eat}.

%%%%%%%%%%%%%%%%%%%%%%%%%%%%%%%%%%%%%%%%%%%%%%%%%%%%%%%%%%

\section{Systematic uncertainties}
\label{sec:uncertainties}

\begin{table}[h!]
\caption{The relative systematic uncertainties of $Y^{\rm{near}}$, $Y^{\rm{away,LM}}$, $F$, $v_{2}$, and $v_{3}$. Numbers given in ranges correspond to minimum and maximum uncertainties. ``negl." is assigned if the systematic deviation is contaminated by the statistical fluctuation. ``N.A" is assigned when the systematic variation is not relevant to the measurement. }
\centering
\label{tab:syst}
\resizebox{\textwidth}{!} {
\begin{tabular}{c|ccccccc}
\hline 
\multirow{3}{*}{Sources}  & \multicolumn{7}{c}{Systematic uncertainty (\%)} \\ \cline{2-8} 
& \multirow{2}{*}{$Y^{\rm{near}}$} & \multirow{2}{*}{$Y^{\rm{away,LM}}$} & \multirow{2}{*}{$F$} & \multicolumn{2}{c}{$v_{2}$} & \multicolumn{2}{c}{$v_{3}$}  \\   \cline{5-8}
& & & & pp & p--Pb & pp & p--Pb  \\ \cline{1-8} 
Primary vertex       & $\pm$0.2--0.5 & $\pm$0.1      & $\pm$1.0--2.5 & $\pm$0.2--1.8 & $\pm$0.8 & $\pm$1.4 & $\pm$3.9 \\ 
Pileup rejection     & $\pm$0.1--0.5 & $\pm$0.2      & $\pm$0.4--1.5 & negl.         & $\pm$0.6 & negl. & $\pm$1.4 \\ 
Tracking		     & $\pm$1.0--3.0 & $\pm$2.0      & $\pm$0.6--2.4 & $\pm$0.2--3.0 & negl. & $\pm$5.0--6.9 & negl. \\ 
Event mixing	     & $\pm$0.2--0.7 & $\pm$0.2--0.5 & $\pm$0.0--3.3 & $\pm$0.3--4.6 & $\pm$0.8 & $\pm$2.8--3.1 & $\pm$0.8 \\ 
LM definition   	 & N.A.          & $\pm$0.5--3.5 & $\pm$0.7--6.0 & negl.         & $\pm$1.9 & negl. & $\pm$9.2\\ 
ITS-TPC matching 	 & $\pm$2.0--3.0 & $\pm$2.0--3.0 & N.A.          & N.A.          & N.A. & N.A. & N.A\\ 
Efficiency correction& $\pm$1.0--4.4 & $\pm$1.0--4.4 & N.A.          & N.A.          & N.A. & N.A. & N.A\\ 
$\eta$ gap range   	 & N.A.          & N.A.          & $\pm$0.1--3.2 & $\pm$1.0--5.0 & $\pm$0.4 & negl. & negl. \\ 

\hline 
Total (in quadrature)& $\pm$2.5--6.1 & $\pm$5.0--5.5 & $\pm$1.8--7.1 & $\pm$0.8--5.7 & $\pm$2.3 & $\pm$6.1--7.5 & $\pm$10.1 \\ 
\hline 
\end{tabular}
}
\end{table}

The systematic uncertainties of $Y^{\rm{near}}$, $Y^{\rm{away,LM}}$, and $F$ are estimated by varying the analysis selection criteria and corrections in pp collisions. The systematic uncertainties of $v_{2}$ and $v_{3}$ are estimated in pp and p--Pb collisions. All systematic uncertainties are summarized in Table~\ref{tab:syst}.

The uncertainty associated to the selected range of the primary vertex is estimated by varying the accepted range from $|z_\mathrm{vtx}|<$~8~cm to $|z_\mathrm{vtx}|<$~6~cm. The variation of the range allows to test acceptance effects on the measurement. The estimated uncertainties of $Y^{\rm{near}}$ and $Y^{\rm{away,LM}}$ are 0.2--0.5\% and 0.1\%, respectively. The uncertainties associated with the primary vertex selection are estimated to be 1.0--2.5\%, 0.2--1.8\%, and 1.4--3.9\% for $F$, $v_{2}$, and $v_{3}$, respectively. 

Another source of systematic uncertainty is related to pileup rejection. Pileup events are rejected with the changed rejection criteria and the number of track contributors required for the reconstruction of pileup event vertices, where the number is changed from 3 to 5. The uncertainties of $Y^{\rm{near}}$ and $Y^{\rm{away,LM}}$ are estimated to be 0.1--0.5\% and 0.2\%, respectively. The estimated uncertainties are negligible for $F$. The estimated uncertainties of $v_{2}$ and $v_{3}$ are 0.6\% and 1.4\%, respectively.

The systematic uncertainty from the track selection criterion is estimated by employing another track selection criteria, denoted global tracks, which are optimized for particle identification. Each global track is required to have at least one SPD hit. Due to inefficient parts of the SPD, the azimuthal distribution of global tracks is not uniform. The estimated uncertainty of observables for $v_{2}$ is 0.2--3.0\% and 5.0--6.9\% for $v_{3}$. 

An additional systematic uncertainty from the event-mixing is estimated by varying the interval of the primary vertex range, where events are mixed. The default value of 2~cm is changed to 1~cm. The resulting uncertainty of jet fragmentation yield ($Y^{\rm{near}}$ and $Y^{\rm{away,LM}}$) is 0.2--0.7\%. The uncertainties of $F$, $v_{2}$, and $v_{3}$ are estimated to be 0.3--4.6\%.

The systematic uncertainty from the LM definition is estimated by changing the range of the multiplicity percentile as the LM is not definitive. The default range for the LM is set to 60--100\%, and changed to 70--100\%. The uncertainty of $Y^{\rm{away,LM}}$ is estimated to be 0.5--3.5\%. Note that the measurement of $Y^{\rm{near}}$ is not relevant to the LM definition, and the uncertainty is not estimated. The uncertainties of $F$, $v_{2}$, and $v_{3}$ are estimated to be 0.7--9.2\%.

The systematic uncertainty from matching the track reconstructed by ITS and TPC is estimated by comparing the matching efficiency evaluated using reconstructed tracks and generated particles. The estimated uncertainties of $Y^{\rm{near}}$ and $Y^{\rm{away,LM}}$ are 2.0--3.0\%.

The systematic uncertainty from the efficiency correction for unidentified charged particles is estimated by comparing correlation functions of true particles with correlation functions of reconstructed tracks with the efficiency correction in the simulation. The estimated uncertainty of observables is 0.1--5.0\%.

Due to the limited $\eta$-acceptance of the TPC, non-flow contributions mainly originating from jet fragmentations affect the flow measurement. As the shape of jet fragmentations is getting broader with the decreasing $\pt$, the systematic uncertainty from $\eta$-acceptance is significantly dependent on the $p_{\rm{T}}$. To estimate the related uncertainty, the minimum $\Delta\eta$ gap is changed from 1.6 to 1.7 for constructing long-range $\Delta\varphi$ correlations.  The estimated uncertainties of $F$, $v_{2}$, and $v_{3}$ are 0.1--5.0\%.








\section{Experimental setup}
\label{sec:experiment}

The analysis is performed using data samples from pp collisions at $\sqrt{s} = 13$~TeV collected between 2016 and 2018, as well as from p--Pb collisions at $\sqrt{s_\mathrm{NN}} = 5.02$~TeV in 2016 during the LHC Run 2 period. A full description of the ALICE detector and its performance during the LHC Run 2 can be found in Refs~\cite{Aamodt:2008zz, Abelev:2014ffa}. The analysis utilizes the V0 detector~\cite{Abbas:2013taa}, the Inner Tracking System (ITS)~\cite{aliceITS}, and the Time Projection Chamber (TPC)~\cite{aliceTPC}. 

The V0 detector consists of two stations on both sides of the interaction point, V0A and V0C, each made of 32 plastic scintillator tiles, covering the full azimuthal angle within the pseudorapidity intervals $2.8 < \eta < 5.1$ and $-3.7 < \eta < -1.7$, respectively. The V0 provides a minimum-bias (MB) trigger in both pp and p--Pb collisions and an additional high-multiplicity trigger in pp collisions. The MB trigger is obtained by a time coincidence of V0A and V0C signals. The charged particle multiplicity is determined based on the sum of the V0A and V0C signals, which is denoted as V0M. The high-multiplicity trigger requires the V0M signal to exceed five times the mean value measured in MB collisions, selecting the 0.1\% of MB events with the largest V0M multiplicity. The analyzed data samples of MB and high-multiplicity pp events at $\sqrt{s}=$~13 TeV correspond to integrated luminosities ($\mathcal{L}_\mathrm{int}$) of 19 nb$^{-1}$ and 11 pb$^{-1}$, respectively~\cite{ALICE-PUBLIC-2016-002}. In p--Pb collisions at $\sqrt{s_\mathrm{NN}} = 5.02$ TeV, the number of events corresponding to $\mathcal{L}_\mathrm{int} = 3$ nb$^{-1}$ is used for the analysis. 

The primary vertex positions are reconstructed from signals measured by the Silicon Pixel Detector (SPD)~\cite{Santoro2009:ALICESPD}, consisting of the two innermost layers of the ITS. The reconstructed primary vertices are required to be within 8 cm of the nominal interaction point along the beam direction. Pileup events are determined and rejected if the longitudinal displacement of the secondary vertex is greater than 0.8 cm in pp collisions. The probability of pileup events is estimated to be about 0.6\% for MB and high-multiplicity events in pp collisions. The pileup probability is estimated to be negligible in p--Pb collisions. 

Charged-particle tracks are reconstructed using the combined information of the ITS and TPC in a uniform magnetic field of 0.5 T along the beam direction by the solenoid. The ITS is a silicon tracker with six layers of silicon sensors. The SPD consists of the two innermost layers. The next two layers are the SDD (Silicon Drift Detector), and the outermost layers are the SSD (Silicon Strip Detector). 
For charged particles emitted from a vertex located within $z_\mathrm{vtx}<8$ cm along the beam direction, the ITS and TPC provide pseudorapidity coverage of $|\eta|<1.4$ and 0.9, respectively. Both detectors have full coverage in azimuth. 
Charged particle tracking is performed using the ITS and TPC, which can reconstruct tracks down to a transverse momentum ($\pt$) of 0.15 GeV/$c$ with an efficiency of approximately 65\%. The efficiency increases to 80\% for intermediate $\pt$ values of 1--5~GeV/$c$. The $\pt$ resolution is approximately 1\% for primary charged particles~\cite{ALICE-PUBLIC-2017-005} with $\pt<$ 1~GeV/$c$, and it linearly increases to 6\% at $\pt\sim$ 50~GeV/$c$ in pp collisions and 10\% in p--Pb collisions~\cite{ALICE:2018vuu}.

The charged-particle-selection criteria are optimized to ensure a uniform efficiency over the sensitive TPC volume to mitigate the effects of small areas where some ITS layers are inactive in both collision systems. The selection consists of two classes of tracks. Those in the first class must have at least one hit in the SPD. Tracks of the second class do not have any hits in the SPD, but their origin is constrained to the primary vertex~\cite{ALICE:2012eyl}. 

\section{Analysis procedure}
\label{sec:ana}
\subsection{Two-particle angular correlations}
Two-particle angular correlations are measured as functions of the relative azimuthal angle ($\Delta\varphi$) and the relative pseudorapidity ($\Delta\eta$) between a trigger and associated particles
\begin{eqnarray}
\frac{1}{N_{\rm{trig}}} \frac{ \rm{d}\it{}^{2} N_{\rm{pair}} }{ \rm{d} \Delta\eta \rm{d}\Delta\varphi} = B(0, 0)\frac{S(\Delta\eta, \Delta\varphi)}{B(\Delta\eta, \Delta\varphi)}  \Big\lvert_{\pttrig,\,\ptassoc},
\label{eq:corrfunction}
\end{eqnarray}
where the transverse momentum range for associated particles ($p_\mathrm{T,assoc}$) is $1<p_\mathrm{T,assoc}<4$~GeV/$c$ for different transverse momentum ranges of trigger particles ($p_\mathrm{T,trig}$).
The numbers of trigger particles and trigger-associated particle pairs are denoted as $N_\mathrm{trig}$ and $N_\mathrm{pair}$, respectively.
The quantity $S(\Delta\eta, \Delta\varphi)$ represents the average number of pairs in the same event, while $B(\Delta\eta, \Delta\varphi)$ denotes the number of pairs in mixed events. The normalization of $B(\Delta\eta, \Delta\varphi)$ is represented by $B(0,0)$. To correct for acceptance effects, $S(\Delta\eta, \Delta\varphi)$ is divided by $B(\Delta\eta, \Delta\varphi)/B(0,0)$. The particles are weighted by the inverse of the tracking efficiency, which is obtained in the same way as in the previous study~\cite{ALICE:2021nir}. In that study, the tracking efficiency was calculated using a detector simulation with the PYTHIA 8 event generator and the GEANT3 transport code~\cite{Brun:1994aa}. To account for differences in particle composition between real data and PYTHIA, the tracking efficiency is determined by re-weighting the primary particle composition based on a data-driven method~\cite{ALICE:2018hza, ALICE:2018vuu}.
%Therefore, this method improves the jet-yield extraction and has no impact on the flow extraction.
The pairs in mixed events are required to have primary vertices within the same 2 cm wide $z_{\rm vtx}$ interval and the correlation functions are averaged over the vertex bins, resulting in the final per-trigger yield~\cite{Kopylov:1974th,Adam:2016tsv}. The lower limit of $p_\mathrm{T,trig}$ and $p_\mathrm{T,assoc}$ ($>$~1~GeV/$c$) is chosen in order to avoid jet-like contributions from lower $p_\mathrm{T}$ particles which extend into the larger $\Delta\eta$ range because of the limited $\eta$ acceptance~\cite{ALICE:2021nir}. 

\begin{figure}[h!]
		\includegraphics[width=0.5 \textwidth]{figures/Fig1_ppHigh.pdf} 
		\includegraphics[width=0.5 \textwidth]{figures/Fig1_ppLow.pdf} 
  		\includegraphics[width=0.5 \textwidth]{figures/Fig1_pPbHigh.pdf}
		\includegraphics[width=0.5 \textwidth]{figures/Fig1_pPbLow.pdf}
\caption{The two-dimensional correlation functions as functions of $\Delta\eta$ and $\Delta\varphi$ are presented for high-multiplicity (0--0.1\% or 0--5\%, on the left) and low-multiplicity (60--100\%, on the right) events in $\sqrt{s}=13$ TeV pp collisions in the top panels. The corresponding distributions for $\sqrt{s_{\mathrm{NN}}}=5.02$ TeV p--Pb collisions are shown in the bottom panels. The intervals of $\pttrig$ and $\ptassoc$ are 1~$<\it{p}_{\rm{T,trig}}<$~2~GeV/$c$ and 1~$<\it{p}_{\rm{T,assoc}}<$~4~GeV/$c$, respectively, in all cases}
\label{fig:doubleridge}
\end{figure}

The two-dimensional correlation functions from $pp$ collisions at $\sqrt{s}=13$ TeV are displayed in the top panels of Fig.~\ref{fig:doubleridge}, while those from $p$--Pb collisions at $\sqrt{s}=5.02$ TeV are presented in the bottom panels. The left column shows the high-multiplicity (0--0.1\% for $pp$ and 0--5\% for $p$--Pb) events, and the right column displays the low-multiplicity (60--100\% for both $pp$ and $p$--Pb) events.  
%The multiplicity percentile of p--Pb collisions is wider as particle multiplicity, ranging from 0--5\%, compared to that of pp collisions.
%The $z$-axis for the correlation yield is properly scaled in order to zoom in the larger $\Delta\eta$ region.
The $z$-axis is scaled in order to resolve the ridge structures at large $\Delta\eta$ regions.
As a result, the jet peaks are sheared off in all figures. The flow modulation structure is clearly observed to emerge in the high-multiplicity collisions for both systems, while it is not seen in the low-multiplicity collisions. The away-side regions are populated mostly by back-to-back jet correlations. 
%but they are reduced and compatible with the one in the near side in $|\Delta\eta| > 1.6$.

The per-trigger yield is determined by integrating the correlation function at large $\Delta\eta$ ($1.6<|\Delta\eta|<1.8$) to remove non-flow contributions from near-side jet fragments.
The per-trigger yield as a function of $\Delta\varphi$ is expressed as:

\begin{eqnarray}
Y(\Delta\varphi) = \frac{1}{N_{\rm{trig}}} \frac{\rm{d}N_{\rm{pair}}}{\rm{d}\Delta\varphi} = \int_{1.6<|\Delta \eta|<1.8} \left(\frac{1}{N_{\rm{trig}}} \frac{\rm{d}^{2}N_{\rm{pair}}}{\rm{d}\Delta\eta \rm{d}\Delta\varphi}\right) \frac{1}{\delta_{\Delta\eta}} \rm{d}\Delta \eta,
\label{eq:pertrigger}
\end{eqnarray}

where the factor $\delta_{\Delta\eta}=$~0.4 is used as a normalization to obtain the per-trigger yield per unit of pseudorapidity.
The per-trigger yield is then extracted for multiple percentile ranges, including 0-0.1\%, 1-5\%, 5-20\%, 20-60\%, and 60-100\% in proton-proton (pp) collisions, and 0-5\%, 5-10\%, 10-20\%, 20-40\%, 40-60\%, and 60-100\% in proton-lead (p-Pb) collisions.  This is done as a function of $p_{\mathrm{T,trig}}/p_{\mathrm{T,assoc}}$ intervals.
%The Zero-Yield-At-Minimum (ZYAM) procedure~\cite{Ajitanand:2005jj} is the method used to subtract the baseline of the correlation. 


\subsection{Extraction of flow coefficients}

As discussed in Refs.\cite{ATLAS:2015hzw,ATLAS:2016yzd}, the correlation function in a given multiplicity percentile is fitted with 
\begin{eqnarray}
\label{eq:narray}
Y_{\rm{HM}}(\Delta\varphi) = G~(1 + 2v_{2,2}\cos(2\Delta\varphi) + 2v_{3,3}\cos(3\Delta\varphi)) + F~Y_{\rm{LM}}(\Delta\varphi),
\end{eqnarray}
where $Y_{\rm{LM}}(\Delta\varphi)$ is the measured per trigger yield from low-multiplicity events. The normalization factor for the first three Fourier terms, which parameterize the long-range, flow-like correlation, is denoted as $G$. The scale factor $F$ compensates for the increased yield of away-side-jet hadrons in a specific multiplicity bin, relative to the low-multiplicity template that corresponds to the 60-100\% percentile~\cite{ALICE:2013tla,ALICE:2014mas}.
The fit determines the scale factor $F$, pedestal $G$, and $v_{n,n}$ and is performed in various high-multiplicity classes as well as in different $p_\mathrm{T,trig}$ and $p_\mathrm{T,assoc}$ intervals. 
%This method does not rely on the zero yield at minimum (ZYAM) hypothesis~\cite{Ajitanand:2005jj} to subtract an assumed flat combinatorial component from the low-multiplicity template as done previously in Refs.~\cite{ATLAS:2012cix,ATLAS:2014qaj}. 
This method assumes that $Y_{\rm{LM}}$ does not contain a peak in the near side that would originate from jet fragmentation, and that the jet shape remains unchanged in high-multiplicity events compared to low-multiplicity events. The first assumption is ensured using the selected low-multiplicity template. The second assumption, which involves the modification of jet shapes, was tested by projecting the near-side jet peaks onto $\Delta\eta$. This projection did not reveal a strong dependence on multiplicity for the jet shape~\cite{LAKOMOV2017329}. However, the effect of the possible modification of the away-side jet yield on the result of the template fit was investigated by modifying the away-side shape by the observed modification between low- and high-multiplicity events. The effect on the extracted $v_2$ is found to be less than 1$\%$ in pp and p--Pb collisions in the kinematic ranges of this analysis. As for $v_3$, the effect is about 3\% and 8$\%$ maximum in pp and p--Pb collisions, respectively.  This modification of the shape is considered one of the sources of systematic uncertainty and will be discussed in Sec.~\ref{sec:uncertainties}.

\begin{figure}[h!]
	\centering
	\hspace{-3em}\includegraphics[width=0.6\textwidth]{figures/Fig1_FlowExt.pdf} 
	\caption{The template fit results with the low-multiplicity templates. The black markers show the signal for the 0--0.1\% multiplicity percentile and its fit as a blue band. The red squares correspond to the low-multiplicity signal. The orange and green curves correspond to the extracted $v_{2,2}$ and $v_{3,3}$ signals, respectively. The fit to the signal is shown as a blue band, and the signal-to-fit ratio is shown in the bottom panel. The $\chi^{2}$ divided by the number of degrees of freedom is 0.894}
	\label{fig:flowext}
\end{figure}

Figure~\ref{fig:flowext} shows the template fit results for the 0--0.1\% multiplicity percentile in pp collisions at $\sqrt{s}$ = 13 TeV. The per-trigger yield for low multiplicity, $Y_{\rm{LM}}(\Delta\varphi)$, is represented by red squares. The values of $v_{2,2}$ and $v_{3,3}$ extracted from the fit are shown in orange and green, respectively. Values of the extracted scale factor $F$ in different multiplicity percentiles and systems are summarized in Tab.~\ref{tab:FpppPb}. In pp collisions, the value of $F$ is observed to slightly increase as the event multiplicity increases, with the highest multiplicity bin having a value approximately 25\% larger than the 20--60\% bin. A similar trend is observed for p--Pb collisions, though with a weaker dependence on multiplicity percentile. Comparing a similar centrality percentile of pp and p--Pb collisions, the value of $F$ in p--Pb collisions is found to be smaller and closer to unity.
%Those are larger at high multiplicity in pp collisions, while they are closer to unity at low multiplicity in p--Pb collisions.
This suggests that the jet fragmentation yield on the away-side increases with multiplicity, which is more pronounced in pp collisions. The difference between the two systems is likely to be explained by the true geometry driven centrality in p--Pb collisions, as opposed to the jet dominated bias in pp collisions.
The previous analyses published by ALICE in Refs. ~\cite{ALICE:2012eyl,ALICE:2013snk} assumed that the jet contribution remains constant as a function of multiplicity (i.e. $F$ was assumed to be 1). However, this assumption may lead to an underestimation of non-flow contamination in the final measurements of anisotropic flow.
%The fit to the signal, Eq.~\ref{eq:narray}, is shown as a blue band, and the signal-to-fit ratio is shown in the bottom panel. 
%let's add chiq value of the fit
\begin{table}[h!]
\caption{The scale factor $F$ for various multiplicity percentiles in pp (top) and p--Pb (bottom) collisions. Note that there are only statistical errors from the default event and track selections.}
%\begin{tabular}{|c|c|c|c|c|c}
%\hline
%V0M (pp)& 0--0.1\% & 1--5\% & 5--20\% & 20--60\% \\
%\hline
%$F$ & 1.504$\pm$0.017 & 1.414$\pm$0.030 & 1.360$\pm$0.019 & 1.208$\pm$0.015 \\
%\hline
%\end{tabular}
\centering
\resizebox{\textwidth}{!}{%
\begin{tabular}{|c|c|c|c|c|c|c|}
\hline
	V0M (pp)& 0--0.1\% & 1--5\% & 5--20\% & 20--60\% & &\\
\hline
	$F$ & 1.504$\pm$0.017 & 1.414$\pm$0.030 & 1.360$\pm$0.019 & 1.208$\pm$0.015 & & \\
\hline
V0A (p--Pb)& 0--5\% & 5--10\% & 10--20\% & 0--20\% & 20--40\% & 40--60\% \\
\hline
$F$& 1.135$\pm$0.026 & 1.140$\pm$0.026 & 1.152$\pm$0.021 & 1.145$\pm$0.017 &1.092$\pm$0.015 & 1.083$\pm$0.015 \\
\hline
\end{tabular}
}
\label{tab:FpppPb}
\end{table}

In the following, the near and away-side jet fragmentation yields are calculated to verify the template fit method by comparing the jet fragmentation yields to the PYTHIA model.
The near-side jet-like yields were extracted from the near-side $\Delta\eta$ correlations, defined in $|\Delta\varphi|<$~1.3 as
\begin{eqnarray}
Y^\mathrm{near}_{\rm{frag}} = \int_{|\Delta \eta|<1.3} \left( \frac{1}{\it{N}_{\rm{trig}}} \frac{ \rm{d}\it{}N_{\rm{pair}} }{ \rm{d}\Delta\eta } \right) \rm{d} \Delta\eta.
\label{eq:Ynear}
\end{eqnarray}
%The range of $\Delta\varphi$ is chosen to be 1.3 in order for the projection range to fully cover the near-side peak in $\Delta\varphi$. 
%The correlations in the same jet mainly contribute to $(\Delta\eta, \Delta\varphi) \sim (0,0)$ due to the similar outgoing direction of particles inside the jet along the jet axis. 
%As can be seen in Eq.~\ref{eq:Ynear}, the yield of the jet fragmentation is calculated by integrating the $\Delta\eta$ correlation over $|\Delta\eta|<1.3$ after applying the ZYAM procedure~\cite{Ajitanand:2005jj} because the flow coefficients have a weak dependence on $\eta$~\cite{ATLAS:2011ah,PHENIX:2018hho,ALICE:2016tlx}. The ZYAM procedure is applied by finding the minimum value of $\Delta\eta$ correlations within $|\Delta\eta|<1.3$, which results in pointing the minimum position to be $|\Delta\varphi|=1.3$.
As flow has a weak $\eta$ dependence~\cite{ATLAS:2011ah,PHENIX:2018hho,ALICE:2016tlx}, the jet fragmentation yield can be calculated after the ZYAM background subtraction~\cite{Ajitanand:2005jj} by integrating the $\Delta\eta$ correlation over $|\Delta\eta|<1.3$, as written in Eq.~\ref{eq:Ynear}. The ZYAM procedure finds the minimum value of $\Delta\eta$ correlations within this range, which due to the shape of the correlation function is at $|\Delta\varphi|=1.3$.

%As can be seen in Eq.~\ref{eq:Ynear}, the per-trigger yield of away-side jet fragments is calculated by integrating the $\Delta\eta$ correlation over $|\Delta\eta|<1.3$ after applying the ZYAM procedure~\cite{Ajitanand:2005jj} because the flow coefficients have a weak dependence on $\eta$~\cite{ATLAS:2011ah,PHENIX:2018hho,ALICE:2016tlx}. The ZYAM procedure is applied by finding the minimum value of $\Delta\eta$ correlations within $|\Delta\eta|<1.3$, which results in pointing the minimum position to be $|\Delta\varphi|=1.3$.

The away-side jet-like yield in data is calculated from the low-multiplicity template fit method as $Y^{\rm{away, HM}}_{\mathrm{frag}} = Y^{\rm{away, LM}}_{\mathrm{frag}} \times F$, where $F$ is the parameter from Eq.~\ref{eq:narray}. The $Y^{\rm{away, LM}}_{\mathrm{frag}}$ is directly obtained by integrating the away-side low-multiplicity $\Delta\varphi$ correlation function in the low-multiplicity sample over $\pi/2 < \Delta\varphi < 3\pi/2$.
It can be seen that while PYTHIA
, it is possible to directly measure  $Y^{\rm{away}}$ from the $\Delta\varphi$ correlation functions since the model does not include any flow contributions.

\begin{figure}[h!]
	\centering
	\hspace{-3em}\includegraphics[width=0.6\textwidth]{figures/Fig5_Plot_v2Mult.pdf} 
	\caption{The $Y^{\rm{frag}}$ for the near- and away-side as a function of multiplicity percentiles with both ALICE and PYTHIA data. The ratio includes the combined statistical and systematic errors in quadrature. ALICE data points are shown as circle and square markers, whereas PYTHIA is shown as lines. The near-side yields are shown in black, and the away-side yields in blue.}
	\label{fig:Ymult}
\end{figure}

Figure~\ref{fig:Ymult} presents the $Y^{\mathrm{near}}_{\rm{frag}}$ and $Y^{\mathrm{away}}_{\rm{frag}}$, for both ALICE data and PYTHIA 8 Tune 4C, as a function of V0M multiplicity percentile in pp collisions at $\sqrt{s}=13$ TeV. The transverse momentum range for trigger particles is $1<p_\mathrm{T,trig}<2$ GeV/c and for associated particles $1<p_\mathrm{T,assoc}<4$ GeV/c.
The near- to away-side ratio for ALICE and PYTHIA data is shown in the bottom panel. While PYTHIA overestimates both near-side and away-side yields measured by ALICE, the ratio to PYTHIA is consistent with the ALICE data in the all considered V0M multiplicity intervals. The value of this ratio can be explained by the pair acceptance effect caused by the limited ALICE $\eta$ acceptance~\cite{PHENIX:2006gto}, which implies that the enhanced jet fragmentation yields in away-side in high-multiplicity events with respect to low-multiplicity events~\cite{ALICE:2013tla,ALICE:2014mas} are taken into account by the low-multiplicity template method. In summary, the difference between the near-side and away-side jet fragmentation yields in PYTHIA is solely caused by the jet acceptance effects of the two-particle correlation functions. This ratio in data where the away-side jet fragmentation yields are measured with the low-multiplicity template agrees well with PYTHIA as well as the calculations in Ref.~\cite{PHENIX:2006gto}.
%It is worthwhile to noting that, even though this method measures the jet yields on the away side from the fit, these yields were not reported in previous measurements~\cite{}.

%\begin{figure}[h!]
%		\includegraphics[width=0.5 \textwidth]{figures/Fig1_ppSub.pdf}
%  		\includegraphics[width=0.5 \textwidth]{figures/Fig1_pPbSub.pdf}
%\caption{The subtracted one as (0--0.1)\%-$F$(60--100\%) is shown on the right. The subtracted one as (0--5)\%-$F$(60--100\%) is shown on the right. Note that the near-side jet peaks exceed the chosen range of the $z$-axis. The intervals of $\pttrig$ and $\ptassoc$ are 1~$<\it{p}_{\rm{T}}<$~2~GeV/$c$ in all cases.}
%\label{fig:doubleridge_sub}
%\end{figure}

%The subtracted one as (0--5)\%-$F$(60--100\%) on the right. Similarly, the ones in p--Pb collisions at $\sqrt{s_{\mathrm{NN}}}=5.02$ TeV are shown in Fig.~\ref{fig:doubleridge_sub}. The $F$ values can be found in Tables~\ref{tab:Fpp} and \ref{tab:Fpb}, which come from the low-multiplicity template fit in a given multiplicity percentile as described in Sec.~\ref{sec:ana}. 
%show an asymmetry in delta eta, e.g. in pp collisions the signal
%rises going from negative to positive delta eta values

The flow coefficients, $v_{n}$, of the trigger particles, can be extracted from the template fit with the use of the observed factorization of $v_{n,n}$ coefficients to single harmonics~\cite{ATLAS:2015hzw,ATLAS:2016yzd} by using
\begin{eqnarray}
v_{n}(p_{\rm{T,trig}}) = v_{n,n}(p_{\rm{T,trig}}, p_{\rm{T,assoc}}) / \sqrt{ v_{n,n}(p_{\rm{T,assoc}},p_{\rm{T,assoc}})},
\end{eqnarray}
where $v_{n,n}(p_{\rm{T,assoc}}, p_{\rm{T,assoc}})$ are measured in 1~$<p_{\rm{T,trig}}<$~4~GeV/$c$ interval as the $p_{\rm{T,assoc}}$ range is fixed with 1~$<p_{\rm{T,assoc}}<$~4~GeV/$c$. In the following sections, unless explicitly stated otherwise, $v_n$ will refer to $v_n(p_{\rm{T,trig}})$.
Different event scale selections were investigated by selecting events that include a hard jet or a high-$\pt$ leading particle within the mid-rapidity region.
%It is possible to study the impact parameter dependence of flow coefficients in pp collisions~\cite{Sjostrand:1986ep,Frankfurt:2010ea}.
This event scale was set by requiring a minimum $\pt$ of the leading track ($\ptlead$) or the reconstructed jet ($\ptjet$) at midrapidity. The leading particle track was required to be within $|\eta|<0.9$ and $0<\varphi<2\pi$, and the jets were reconstructed with the anti-$k_{\rm{T}}$ algorithm~\cite{Cacciari:2008gp,Cacciari:2011ma}, with $R=0.4$ using charged particles only. Jet constituents were combined using the boost invariant $\pt$ recombination scheme. In the same way, as for the leading particle tracks, the jets are selected in the full azimuth ($0<\varphi<2\pi$) but in an $\eta$-range of $|\eta_\mathrm{jet}|<0.4$. The $\pt$ of jets $\ptjet$ is corrected for the underlying event density that is measured using the $k_{\rm{T}}$ algorithm with $R=$~0.2 following the procedure described in Ref.~\cite{Acharya:2018eat}.

%%%%%%%%%%%%%%%%%%%%%%%%%%%%%%%%%%%%%%%%%%%%%%%%%%%%%%%%%%

\section{Systematic uncertainties}
\label{sec:uncertainties}

\begin{table}[h!]
\caption{The relative systematic uncertainties of $Y^{\rm{near}}$, $Y^{\rm{away,LM}}$, $F$, $v_{2}$, and $v_{3}$. The quoted ranges correspond to minimum and maximum uncertainties. Those uncertainties that are considered to be negligible are marked ``negl.". The systematic variations which are not relevant for the measurement are denoted as ``N.A".}
\centering
\label{tab:syst}
\resizebox{\textwidth}{!} {
\begin{tabular}{c|ccccccc}
\hline 
\multirow{3}{*}{Sources}  & \multicolumn{7}{c}{Systematic uncertainty (\%)} \\ \cline{2-8} 
& \multirow{2}{*}{$Y^{\rm{near}}$} & \multirow{2}{*}{$Y^{\rm{away,LM}}$} & \multirow{2}{*}{$F$} & \multicolumn{2}{c}{$v_{2}$} & \multicolumn{2}{c}{$v_{3}$}  \\   \cline{5-8}
& & & & pp & p--Pb & pp & p--Pb  \\ \cline{1-8} 
Primary vertex       & $\pm$0.2--0.5 & $\pm$0.1      & $\pm$1.0--2.5 & $\pm$0.2--1.8 & $\pm$0.8 & $\pm$1.4 & $\pm$3.9 \\ 
Pileup rejection     & $\pm$0.1--0.5 & $\pm$0.2      & $\pm$0.4--1.5 & negl.         & $\pm$0.6 & negl. & $\pm$1.4 \\ 
Tracking		     & $\pm$1.0--3.0 & $\pm$2.0      & $\pm$0.6--2.4 & $\pm$0.2--3.0 & negl. & $\pm$5.0--6.9 & negl. \\ 
Event mixing	     & $\pm$0.2--0.7 & $\pm$0.2--0.5 & $\pm$0.0--3.3 & $\pm$0.3--4.6 & $\pm$0.8 & $\pm$2.8--3.1 & $\pm$0.8 \\ 
Low-mult. definition & N.A.          & $\pm$0.5--3.5 & $\pm$0.7--6.0 & negl.         & $\pm$1.9 & negl. & $\pm$9.2\\ 
ITS-TPC matching 	 & $\pm$2.0--3.0 & $\pm$2.0--3.0 & N.A.          & N.A.          & N.A. & N.A. & N.A\\ 
Efficiency correction& $\pm$1.0--4.4 & $\pm$1.0--4.4 & N.A.          & N.A.          & N.A. & N.A. & N.A\\ 
$\eta$ gap range   	 & N.A.          & N.A.          & $\pm$0.1--3.2 & $\pm$1.0--5.0 & $\pm$0.4 & negl. & negl. \\ 
jet shape change   	 & N.A.          & N.A.          & N.A &  $\pm$1.0& $\pm$1.0 & $\pm$3.0 & $\pm$8.0 \\ 
\hline 
%Total (in quadrature)& $\pm$2.5--6.1 & $\pm$5.0--5.5 & $\pm$1.8--7.1 & $\pm$0.8--5.7 & $\pm$2.3 & $\pm$6.1--7.5 & $\pm$10.1 \\ 
Total (in quadrature)& $\pm$2.5--6.1 & $\pm$5.0--5.5 & $\pm$1.8--7.1 & $\pm$1.3--5.8 & $\pm$2.5 & $\pm$6.8--8.0 & $\pm$12.8 \\ 
\hline 
\end{tabular}
}
\end{table}

Systematic uncertainties are estimated by varying the analysis selection criteria and corrections. Independent systematic checks are performed, and the differences between measured values from each variation and the default selection are considered as the systematic uncertainty for each source. The total systematic uncertainty is obtained by adding the contributions from different sources in quadrature. A summary of all systematic uncertainties is provided in Table~\ref{tab:syst}. 

The uncertainty attributed to the chosen primary vertex range was estimated by varying the accepted range from $|z_\mathrm{vtx}|<$~8~cm to $|z_\mathrm{vtx}|<$~6~cm. The variation of the range allows testing detector acceptance effects on the measurement. The estimated uncertainties of $Y^{\rm{near}}$ and $Y^{\rm{away,LM}}$ are 0.2--0.5\% and 0.1\%, respectively. The uncertainties associated with the primary vertex selection were estimated to be 1.0--2.5\%, 0.2--1.8\%, and 1.4--3.9\% for $F$, $v_{2}$, and $v_{3}$, respectively. 

Another source of systematic uncertainty is related to pileup rejection. Pileup events were rejected with different rejection criteria, such as the number of track contributors required for reconstruction of pileup event vertices, where the number is changed from the default value of 3 to 5. The uncertainties of $Y^{\rm{near}}$ and $Y^{\rm{away,LM}}$ are estimated to be 0.1--0.5\% and 0.2\%, respectively. The estimated uncertainties are negligible for $F$. The estimated uncertainties of $v_{2}$ and $v_{3}$ are 0.6\% and 1.4\%, respectively.

The systematic uncertainty due to choice of track selection criteria was estimated by employing alternative track selection criteria so called “global track”, which are described in Ref~\cite{ALICE:2021ptz}. A global track is required to have two hits in the ITS (at least one in the SPD) and at least 70 clusters in the TPC. Due to inefficiencies in certain parts of the SPD, the azimuthal distribution of global tracks is not uniform. This can be corrected by using corresponding mixed events and accounting for the corresponding tracking efficiency, including the different $p_{\mathrm{T}}$ inefficiencies of this tracking. The systematic uncertainty from the different track selection criteria is  0.2--3.0\% for $v_{2}$ and  5.0--6.9\% for $v_{3}$. 


An additional systematic uncertainty from the event-mixing is estimated by varying the interval of the primary vertex range, where events are mixed. The default value of 2~cm is changed to 1~cm. The resulting uncertainty of jet fragmentation yield ($Y^{\rm{near}}$ and $Y^{\rm{away,LM}}$) is 0.2--0.7\%. The uncertainties of $F$, $v_{2}$, and $v_{3}$ are estimated to be 0.3--4.6\%.

The systematic uncertainty associated with the low-multiplicity definition is estimated by changing the range of the low-V0M-multiplicity percentile. There is no universal definition for the low-multiplicity. The default range for the low-multiplicity in the present paper is set to 60--100\%, and changed to 70--100\%. The uncertainty of $Y^{\rm{away,LM}}$ is estimated to be 0.5--3.5\%. Note that for the measurement of $Y^{\rm{near}}$, the low-multiplicity definition is irrelevant and the corresponding uncertainty is not estimated. The uncertainties of $F$, $v_{2}$, and $v_{3}$ are estimated to be 0.7--9.2\%.

The systematic uncertainty from matching the track reconstructed by the TPC and the corresponding signal in the ITS is estimated by evaluating the fraction of the mismatch between them. The estimated uncertainties of $Y^{\rm{near}}$ and $Y^{\rm{away,LM}}$ are 2.0--3.0\%.

The systematic uncertainty from the efficiency correction for unidentified charged particles is estimated by comparing two correlation functions. One is constructed using true information in MC samples. The other is constructed using the reconstructed tracks, where reconstructed tracks are corrected for tracking efficiency. The estimated uncertainty is 0.1--5.0\%.

Due to the limited $\eta$ acceptance of the TPC, non-flow contributions, mainly originating from jet fragmentations, affect the flow measurement. As the shape of short-range correlations mostly attributed to jets is getting broader with the decreasing $\pt$, the systematic uncertainty from $\eta$-acceptance significantly depends on the $p_{\rm{T}}$. To estimate the related uncertainty, the minimum $\Delta\eta$ gap is changed from 1.6 to 1.7 for constructing the long-range $\Delta\varphi$ correlations.  The estimated uncertainties of $F$, $v_{2}$, and $v_{3}$ are 0.1--5.0\%.

Finally, it is worth considering the possible impact of the multiplicity dependence of the jet shape modifications discussed in Section~\ref{sec:ana}. This is studied by examining the shape modification of the jet peak distribution in the near-side $\Delta\eta$ region as a function of  multiplicity. The observed change in width is used to estimate a possible effect on the distribution in $\varphi$. The effect on $v_2$ is found to be less than 1$\%$ in pp and p--Pb collisions for the kinematic ranges analyzed. For $v_3$, the effect is found to be no more than about 8$\%$ in p--Pb collisions. These values are included in the total systematic uncertainty. However, it is important to note that other analyses with different kinematic ranges should also perform a similar check on the systematics of this effect. It is possible that this effect may not always be as small as in our analysis.






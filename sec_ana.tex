

\section{Experimental setup}
\label{sec:experiment}

The analysis is carried out with data samples of pp collisions at $\sqrt{s} = 13$~TeV collected from 2016 to 2018 during the LHC Run 2 period. The full description of the ALICE detector and its performance in the LHC Run 2 can be found in~\cite{Aamodt:2008zz,Abelev:2014ffa}. The present analysis utilizes the V0~\cite{Abbas:2013taa}, the Inner Tracking System (ITS)~\cite{aliceITS}, and the Time Projection Chamber (TPC)~\cite{aliceTPC} detectors.

The V0 detector consists of two stations placed on both sides of the interaction point, V0A and V0C, each made of 32 plastic scintillator tiles, covering the full azimuthal angle within the pseudorapidity intervals $2.8 < \eta < 5.1$ and $-3.7 < \eta < -1.7$, respectively. The V0 is used to provide a minimum bias (MB) and a high-multiplicity (HM) trigger. The minimum bias trigger is obtained by a time coincidence of V0A and V0C signals. The charged particle multiplicity selection is done on the sum of the V0A and V0C signals, which is denoted as V0M. The high-multiplicity trigger requires that the V0M signal exceeds 5 times the mean value measured in minimum bias collisions, selecting the 0.1\% of MB events that have the largest V0 multiplicity. The analyzed data samples of minimum bias and high-multiplicity pp events at $\sqrt{s}=$~13 TeV correspond to integrated luminosities of 19 nb$^{-1}$ and 11 pb$^{-1}$, respectively~\cite{ALICE-PUBLIC-2016-002}.

The primary vertex position is reconstructed from the measured signals in the Silicon Pixel Detector (SPD), which forms the innermost two layers of the ITS. Reconstructed primary vertices of selected events are required to be located within 8~cm from the center of the detector along the beam direction. The probability of pileup events is about 0.6\% in MB events. Pileup events can be resolved and are rejected if the longitudinal displacement of their primary vertices is larger than 0.8 cm.

Charged-particle tracks are reconstructed by the ITS and TPC, which are operated in a uniform solenoidal magnetic field of 0.5~T along the beam direction. The ITS is a silicon tracker with six layers of silicon sensors where the SPD~\cite{Santoro2009:ALICESPD} comprises the two innermost layers, the next two layers called the Silicon Drift Detector (SDD), and the outermost layers named the Silicon Strip Detector (SSD). The ITS and TPC, covering the full azimuthal range, have acceptances up to $|\eta| < 1.4$ and 0.9, respectively, for detection of charged particles emitted within 8 cm from the primary vertex position ($z_\mathrm{vtx}$) along the beam direction. The tracking of charged particles is done with the combined information of the ITS and TPC that enables the reconstruction of tracks down to 0.15~GeV/$c$, where the efficiency is about 65\%. The efficiency reaches 80\% for intermediate $\pt$, 1 to 5~GeV/$c$. The $\pt$ resolution is around 1\% for primary tracks with $\pt<$~1~GeV/$c$, and linearly increases up to 6\% at $\pt \sim$ 40~GeV/$c$~\cite{Contin_2012:ITSPTRES}.
%https://arxiv.org/pdf/1910.14400.pdf %https://arxiv.org/abs/1402.4476

The charged particle selection criteria are optimized to make the efficiency uniform over the full TPC volume to mitigate the effect of small regions where some of the ITS layers are inactive. The selection consists of two track classes. Those belonging to the first class are required to have at least one hit in the SPD. Tracks from the second class do not have any SPD associated hit and their initial point is instead constrained to the primary vertex~\cite{Adam:2015ewa}.


\section{Analysis procedure}
\label{sec:ana}

The two-particle correlation function is measured as a function of the relative pseudorapidity ($\Delta\eta$) and the azimuthal angle difference ($\Delta\varphi$) between the trigger and the associated particles,
\begin{eqnarray}
\frac{1}{N_{\rm{trig}}} \frac{ \rm{d}\it{}^{2} N_{\rm{pair}} }{ \rm{d} \Delta\eta \rm{d}\Delta\varphi} = B(0, 0)\frac{S(\Delta\eta, \Delta\varphi)}{B(\Delta\eta, \Delta\varphi)}  \Big\lvert_{\pttrig,\,\ptassoc}\quad , 
\label{eq:corrfunction}
\end{eqnarray}
where $\pttrig$ and $\ptassoc$ ($\pttrig>\ptassoc$) are the transverse momenta of the trigger and associated particles, respectively, $N_\mathrm{trig}$ is the number of trigger particles, and $N_\mathrm{pair}$ is the number of trigger-associated particle pairs. The average number of pairs in the same event and in mixed events are denoted as $S(\Delta\eta, \Delta\varphi)$ and $B(\Delta\eta, \Delta\varphi)$, respectively. Normalization of $B(\Delta\eta, \Delta\varphi)$ is done with its value at $\Delta\eta$ and $\Delta\varphi = 0$, represented as $B (0,0)$. Acceptance effects are corrected by dividing $S(\Delta\eta, \Delta\varphi)$ with $B(\Delta\eta, \Delta\varphi)/B (0,0)$. The right-hand side of Eq.~(1) is corrected for the track reconstruction efficiency, which is mainly relevant for the associated particles, as a function of $\pt$ and pseudorapidity. Primary vertices of events to be mixed are required to be within the same, 2 cm wide, $z_{\rm{vtx}}$ interval~\cite{KOPYLOV1974472:evtmixing,Adam:2016tsv} for each multiplicity class. The final per-trigger yield is constructed by averaging correlation functions over these primary vertex bins.

Ridge yields at large $\Delta\eta$ are extracted for various multiplicity classes and $\pt$ intervals. The large $\Delta\eta$ range is selected as $1.6<|\Delta\eta|<1.8$, which is the range where the tracking quality -- efficiency and precision -- is the best. The ridge yield is only reported for $\pt>$~1~GeV/$c$. Below 1~GeV/$c$, the jet-like contribution to the correlation function extends into the region where the ridge yield is measured, 1.6~$<|\Delta\eta|<$~1.8. In this region, the $\Delta\varphi$ distribution, or the so-called per-trigger yield, is expressed as
\begin{eqnarray}
\frac{1}{N_{\rm{trig}}} \frac{ \rm{d}\it{}N_{\rm{pair}} }{ \rm{d}\Delta\varphi } = \int_{1.6<|\Delta \eta|<1.8} \left( \frac{1}{\it{N}_{\rm{trig}}} \frac{ \rm{d}\it{}^{2} N_{\rm{pair}} }{ \rm{d}\Delta\eta d\Delta\varphi} \right) \dfrac{1}{\delta_{\Delta\eta}} \rm{d}\Delta \eta - C_{\rm{ZYAM}}\quad ,
\end{eqnarray}
where $\delta_{\Delta\eta}=$~0.4 is the normalization factor to get the per-trigger yield per unit of pseudorapidity. 

The baseline of the correlation is subtracted by means of the Zero-Yield-At-Minimum (ZYAM) procedure~\cite{Ajitanand:2005jj}. The minimum yield $(C_{\rm{ZYAM}})$ at $\Delta\varphi=\Delta\varphi_{\rm{min}}$ in the $\Delta\varphi$ projection (note that the value of $\Delta\varphi_{\rm{min}}$ can be different in data and in models) is obtained from a fit function, which fits the data with a Fourier series up to the third harmonic. By construction, the yield at $\Delta\varphi_{\rm{min}}$ is zero after subtracting $C_{\rm{ZYAM}}$ from the $\Delta\varphi$ projection. The ridge yield ($Y^{\rm{ridge}}$) is obtained by integrating the near-side peak of the $\Delta\varphi$ projection over $|\Delta\varphi|<|\Delta\varphi_{\rm{min}}|$ after the ZYAM procedure,
\begin{eqnarray}
Y^{\rm{ridge}} = \int_{|\Delta \varphi| < |\Delta\varphi_{\rm{min}}| } \frac{1}{\it{N}_{\rm{trig}}} \frac{ \rm{d}\it{}N_{\rm{pair}} }{ \rm{d}\Delta\varphi }  \rm{d} \Delta\varphi .
\end{eqnarray}

The ridge yield is further studied in events having a hard jet or a high-$\pt$ leading particle in the mid-rapidity region. This event scale is set by requiring a minimum $\pt$ of the leading track ($\ptlead$) or the reconstructed jet ($\ptjet$) at midrapidity. The leading track is selected within $|\eta|<0.9$ and the full azimuthal angle. Jets are reconstructed with charged particles only (track-based jets) with the anti-$k_{\rm{T}}$ algorithm~\cite{Cacciari:2008gp,Cacciari:2011ma} and the resolution parameter $R=$~0.4. The recombination scheme used in this article is the $\pt$ scheme. Jets are selected in $|\eta_\mathrm{jet}|<0.4$ and in the full azimuthal angle. The $\pt$ of jets $\ptjet$ is corrected for the underlying event density that is measured using the $k_{\rm{T}}$ algorithm with $R=$~0.2~\cite{Acharya:2018eat}. 

To quantify the variation of the near-side jet-like peak with event-scale selections with a minimum $p_\mathrm{T,\,LP}$ or $\ptjet$, the near-side jet-like peak yield is extracted from the near-side $\Delta\eta$ correlations. The near-side is defined as $|\Delta\varphi|<$~1.28, where the correlation function is projected on the $\Delta\eta$ axis. The projection range, 1.28, is chosen to fully cover $\Delta\varphi_{\rm{min}}$. The near-side $\Delta\eta$ correlations are then constructed as
\begin{eqnarray}
\frac{1}{N_{\rm{trig}}} \frac{ \rm{d}\it{}N_{\rm{pair}} }{ \rm{d}\Delta\eta } = \int_{|\Delta \varphi|<1.28} \left( \frac{1}{\it{N}_{\rm{trig}}} \frac{ \rm{d}\it{}^{2} N_{\rm{pair}} }{ \rm{d}\Delta\eta d\Delta\varphi} \right) \dfrac{1}{\delta_{\Delta\varphi}} \rm{d} \Delta \varphi - \it{D}_{\rm{ZYAM}} \quad,
\end{eqnarray}
where $\delta_{\Delta\varphi}=$~2.56 is the normalization factor to get per-trigger yield per unit of azimuthal angle.
The minimum yield $(D_{\rm{ZYAM}})$ of the $\Delta\eta$ correlations is found within $|\Delta\eta|<$~1.6 and used for the subtraction from  the $\Delta\eta$ correlations, which results in zero-yield at the minimum. The near-side jet-like peak yield ($Y^{\mathrm{near}}$) is measured by integrating the $\Delta\eta$ correlations over $|\Delta\eta|<$~1.6,
\begin{eqnarray}
Y^{\rm{near}} = \int_{|\Delta \eta|<1.6} \left( \frac{1}{\it{N}_{\rm{trig}}} \frac{ \rm{d}\it{}N_{\rm{pair}} }{ \rm{d}\Delta\eta } \right) \rm{d} \Delta\eta
\end{eqnarray}

\section{Systematic uncertainties of the measured yields}
\label{sec:uncertainties}

The systematic uncertainties of $Y^{\rm{ridge}}$ and $Y^{\rm{near}}$ are estimated by varying the analysis selection criteria and corrections and are summarized in Tab.~\ref{tab:syst}.

\begin{table}[h!]
\caption{The relative systematic uncertainty of $Y^{\rm{ridge}}$ and $Y^{\rm{near}}$. Numbers given in ranges correspond to minimum and maximum uncertainties.}
\centering
\begin{tabular}{c|cc}
\hline 
\multirow{2}{*}{Sources}  & \multicolumn{2}{c}{Systematic uncertainty (\%)} \\\cline{2-3} 
         & $Y^{\rm{ridge}}$ & $Y^{\rm{near}}$ \\ \hline 
Pileup rejection    	& $\pm$0.8--3.9    &$\pm$0.2--2.2	\\ 
Primary vertex	        & $\pm$0.5--2.4	   &$\pm$1.1	\\ 
Tracking		        & $\pm$2.0--4.0    &$\pm$1.5--3.4	\\ 
ZYAM		        	& $\pm$2.1--5.1	   &$\pm$2.2--4.8	\\ 
Event mixing	    	& $\pm$1.0--4.4	   &$\pm$0.5--1.7	\\ 
Efficiency correction	& $\pm$2.5 	    &$\pm$3.1	\\  
Jet contamination   	& $-$18.8--25.9 ($\pt<$~2~GeV/$c$)	&N.A.	\\ \hline 
Total (in quadrature)			& $^{+\rm{4.9}\textrm{--}\rm{9.4}}_{-\rm{19.4}\textrm{--}\rm{21.0}}$ & $\pm$3.9--7.3 \\ 
\hline 
\end{tabular}
\label{tab:syst}
\end{table}

The systematic uncertainties are independent of the event-scale selection except for $D_{\rm{ZYAM}}$ (see below), as expected, since the multiplicity is weakly dependent on the event scale and the ALICE detector is optimized for much higher multiplicities (Pb--Pb collisions), this is in agreement with our expectations.

The uncertainty associated to the pileup rejection is estimated by measuring the changes of results with different rejection criteria from the default one. It is mainly estimated by varying the minimal number of track contributors required for reconstruction of pileup event vertices from 3 to 5. The estimated uncertainty of $Y^{\rm{ridge}}$ is 0.8-3.9\%. The corresponding uncertainty of $Y^{\rm{near}}$ is estimated to be 0.2--2.2\%.
 
Another source of systematic uncertainty is related to the selected range of the primary vertex. The accepted range is changed from $|z_\mathrm{vtx}|<$ 8 cm to $|z_\mathrm{vtx}|<$ 6 cm. The narrower primary vertex selection allows one to test acceptance effects on the measurement. The estimated uncertainty of $Y^{\rm{ridge}}$ is 0.5--2.4\%. The uncertainty for $Y^{\rm{near}}$ is estimated to be 1.1\%.

An additional source of systematic uncertainty is related to the track selection criteria. The corresponding uncertainty is estimated by employing other track selection criteria, denoted global tracks, which are optimized for particle identification. The selection criteria of the global tracks are almost identical to the hybrid tracks. Each global track is required to have at least one SPD hit. Due to inefficient parts of the SPD, the azimuthal distribution of global tracks is not uniform.
The uncertainties associated with the track selection are estimated to be 2.0--4.0\% and 1.5--3.4\% for $Y^{\rm{ridge}}$ and $Y^{\rm{near}}$, respectively.

The systematic uncertainty of $Y^{\rm{ridge}}$ resulting from the ZYAM procedure is estimated by varying the range of the fit, which is used to find the minimum, from $|\Delta\varphi|<\pi/$2 down to $|\Delta\varphi|<1.2$. The estimated uncertainty of $Y^{\rm{ridge}}$ is 2.1--5.1\%. The corresponding uncertainty on $Y^{\rm{near}}$ is estimated by varying the range from $|\Delta\eta|<$~1.6 to $|\Delta\eta|<$~1.5 and 1.7. The estimated uncertainty of $Y^{\rm{near}}$ is 2.2\% for the unbiased case and increases to 4.8\% for the largest event-scale selections. This is the only systematic uncertainty for which a significant dependence on the event scale is observed, reflecting a non-negligible dependence of the near-side magnitude and shape on the event-scale selection.

The source of systematic uncertainty is associated to the choice of the width of $z_{\rm vtx}$ bins that are used in the event mixing method. The default value of 2\,cm is changed to 1\,cm. The resulting uncertainty of $Y^{\rm{ridge}}$ is 1.0--4.4\%.
The uncertainty for $Y^{\rm{near}}$ is about 0.5--1.7\%. The uncertainty from the efficiency correction for charged particles is estimated by comparing correlation functions of true particles with correlation functions of reconstructed tracks with the efficiency correction in simulation. The estimated uncertainties are 2.5\% and 3.1\% for $Y^{\rm{ridge}}$ and $Y^{\rm{near}}$, respectively. 

In the limited $\eta$-acceptance of ALICE, the ridge structure is not flat in $\Delta\eta$ suggesting that jet-like correlations (non-flow) could contribute, implying that they would impact the ridge-yield extraction. We stress that the models used for comparisons also contain such a non-flow effect, but differences in jet-like correlations between data and MC models could influence the interpretation. To account for the related uncertainty, the variation of the yield with $\Delta\eta$ between 1.5 and 1.8, which should be an upper limit of the residual jet-like contamination, is used as a systematic uncertainty of the ridge yield. The estimated upper limit of the uncertainty is $-$25.9\% for the 1.0~$<\pt<$~1.5~GeV/$c$ range,  $-$18.8\% for the 1.5~$<\pt<$~2.0 GeV/$c$ range,  $-$18.9\% for the 1.0~$<\pt<$~2.0~GeV/$c$ range, and negligible for $\pt>$~2.0~GeV/$c$. This uncertainty is considered only for the measured ridge yields.

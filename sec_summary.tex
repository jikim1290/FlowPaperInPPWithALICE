% !TEX root = paper.tex

\section{Conclusions}
\label{sec:summary}
Long-range angular correlations for pairs of charged particles are studied in ${\sqrt{{\textit s}}}=13$~TeV pp and ${\sqrt{{\textit s_{NN}}}}=5$~TeV p--Pb collisions. Fourier coefficients extracted from the long-range correlations ($\Delta\eta > 1.6$) from high to very low multiplicity classes are extracted from the low-multiplicity template fit method which allows one to subtract the enhanced away-side jet yields in high-multiplicity with respect to low-multiplicity events.
%Verification of the non-flow subtraction, demonstrating the importance of publishing the extracted jet yields (data and theory) together with flow results
%Testing lower limit of event multiplicity on flow signal both for pp and p--Pb
%Comparison to different experiment, suggesting a possible apple-to-apple comparison
These results are consistent with ATLAS, and $v_n$ becomes larger for higher $p_\mathrm{T}$, and maximum at $2.5<p_\mathrm{T}<3.0$ GeV/c. 
Additionally, they show a weak multiplicity dependence on $v_2$ both for pp and p--Pb collisions, and a hint of disappearing $v_2$ signal below $N_{ch}$ = 10 in $|\eta|<0.5$. 
The comparisons to viscous hydrodynamic models show that the magnitudes of $v_2$ and their multiplicity dependence are not described by state-of-art hydrodynamic calculations with two initial state models especially for low multiplicity p--Pb and pp collisions. This suggests that the details of the initial state which are more important with decreasing multiplicity~\cite{Greif:2017bnr,Moreland:2018gsh} should be further improved based on experimental measurements. 
%the inclusion of nucleon substructure and the large difference in the preferred pre-equilibrium :Moreland:2018gsh
%free-streaming time determined by the two studies
Furthermore, the events with hard probes such as jets or leading particles do not show any changes in $v_2$ within the uncertainties. 
Even though it would be interesting to compare these results to EPOS LHC and PYTHIA8 String Shoving models as done in \cite{ALICE:2012eyl} for the ridge yields, it is not possible to reliably extract flow coefficients because these models exhibit flow or ridge signals in low-multiplicity events, thus making the use of a low-multiplicity template unfeasible. We leave this comparison for future research work.


% !TEX root = paper.tex

\section{Conclusions}
\label{sec:summary}
Long-range angular correlations for pairs of charged particles are studied in pp collisions at ${\sqrt{{\textit s}}}=13$~TeV and p--Pb collisions at $\sqrt{s_\mathrm{NN}} = 5.02$~TeV. Flow coefficients are extracted from long-range correlations ($1.6 <|\Delta\eta| < 1.8$) for a broad range of event charged-particle multiplicity classes using the template method, which allows one to subtract the enhanced away-side jet fragmentation yields in high-multiplicity events with respect to low-multiplicity events. The method that was used to measure the flow coefficients within the considered kinematic ranges has been verified to be stable.
The systematic uncertainties on $v_2$ and $v_3$ measurements which reflect the possible differences in the away-side jet peak shapes in high- and low-multiplicity events were found to be 1\% and 3–8\%, respectively.
However, it is important that these systematic uncertainties are reevaluated, when analyzing different kinematic ranges, as the effect may not always be negligible.
The measured $p_\mathrm{T}$ dependence of $v_2$ and $v_3$ is consistent with the measurements by ATLAS, and shows that both $v_2$ and $v_3$ increase with $p_\mathrm{T}$ and reach their maximum at $2.5<p_\mathrm{T}<3.0\,\mathrm{GeV}/c$.
The measurement of $v_2$ as a function of charged-particle multiplicity in $|\eta|<0.5$ shows a weak multiplicity dependence both for pp and p--Pb collisions and tends to decrease toward lower multiplicities.
%In pp collisions the results reveal a hint of a disappearing $v_2$ signal below $N_{\mathrm{ch}} = 10$. 
The pp data suggests that the $v_2$ signal may disappear when the measurement is pursued further below $N_{\mathrm{ch}} = 10$. 
%However, it cannot be definitively excluded whether there is a complete disappearance of the $v_2$ signal below this threshold due to the limited statistical precision of the data.

The comparisons to viscous hydrodynamic models show that the magnitudes of $v_2$ and their multiplicity dependence are not described by state-of-the-art hydrodynamic calculations, which simulated initial conditions with two initial state models, especially for low-multiplicity p--Pb and pp collisions. As initial state effects tend to be more important at low multiplicity~\cite{Greif:2017bnr,Moreland:2018gsh}, these results may help to constrain the modeling of the initial state.
Furthermore, the events including hard probes such as jets or high-$p_\mathrm{T}$ leading particles do not show any changes both in $v_2$ and $v_3$ within the uncertainties, which implies that the long-range correlation of soft particles is not significantly modified by the presence of the hard-scattering process. Even though it would be interesting to compare these results to the EPOS LHC~\cite{Pierog:2013ria} and PYTHIA8 String Shoving models~\cite{Bierlich:2017vhg,Bierlich:2019ixq} as done in Ref.~\cite{ALICE:2012eyl} for the ridge yields, it is not possible to reliably extract the flow coefficients because these models exhibit a near-side ridge structure in low-multiplicity events, thus making the use of the low-multiplicity template ill defined~\cite{Ji:2023eqn}.


% !TEX root = paper.tex

\section{Introduction}
\label{sec:intro}

High-energy nucleus--nucleus (AA) collisions exhibit strong collectivity, which has been observed through anisotropy in the momentum distribution of emitted final-state particles at RHIC~\cite{Adams:2005dq,Adcox:2004mh,Arsene:2004fa,Back:2004je} and the LHC~\cite{Abelev:2012di, Abelev:2014pua, ATLAS:2011ah}. This momentum anisotropy is developed by pressure-driven expansion of the strongly interacting quark--gluon plasma (QGP) formed from the initial spatial anisotropy in such collisions.
%The collective effects originate from the pressure-driven expansion of the strongly interacting quark-gluon plasma (QGP) that is formed at the extreme temperatures created in a collision event. 
% [Keep it?]
% This expansion is described by relativistic hydrodynamics~\cite{Romatschke:2007mq,Jeon:2015dfa,Romatschke:2017ejr}, for which the measured anisotropy acts as constraints for the transport properties of the QGP. The understanding of the transport properties has recently been significantly advanced through efforts in experimental~\cite{ALICE:2016kpq,Acharya:2017gsw,Acharya:2017zfg,Acharya:2020taj,ALICE:2021klf,ALICE:2021adw} and theoretical studies~\cite{Niemi:2015qia,Bernhard:2016tnd,Bernhard:2019bmu,Parkkila:2021tqq,Parkkila:2021yha}.
The collective nature of the momentum anisotropy is mostly estimated via long-range particle correlations over a wide range of pseudorapidity. The collective motion of the emitted particles, and thus the medium, is generally quantified with the use of Fourier-series expansion, characterizing the so-called ``anisotropic flow''\cite{Ollitrault:1992bk}. Over the past years, long-range correlations have been also observed in smaller collision systems, particularly in high-multiplicity proton--proton (pp)~\cite{ATLAS:2015hzw,Khachatryan:2015lva,Khachatryan:2016txc,Acharya:2019vdf,ATLAS:2017rtr}, proton--nucleus (pA)~\cite{ALICE:2012eyl,ATLAS:2014qaj,ATLAS:2016yzd,Khachatryan:2016ibd}, and light AA collisions~\cite{PHENIX:2018lia,Aidala:2017ajz}. These observations raise the question of to what extent the small system collisions share a similar underlying mechanism in developing the correlations as heavy AA collisions.
A crucial evidence of a strongly interacting medium in small collision systems would be the presence of jet quenching~\cite{Gyulassy:1990ye,Wang:1991xy}. However, this phenomenon has not yet observed in either high-multiplicity pp or p--Pb collisions~\cite{Adam:2014qja,Khachatryan:2016odn,Adam:2016jfp,Adam:2016dau,Acharya:2017okq}. %A study with two-particle angular correlations in a short range around $(\Delta\eta$, $\Delta\varphi)=(0,0)$ is a good tool for studying jet fragmentation~\cite{Adam:2016tsv}.

In the absence of an inclusive technique in computing the evolution of an out-of-equilibrium strongly coupled quantum chromodynamic medium, the evolution of the soft-sector of AA collisions is separated into multiple stages, and each stage is described by an effective theory. To this date, the combination of color-glass condensate effective field theory (CGC-EFT)~\cite{Schenke:2012wb,Schenke:2012hg}, causal hydrodynamics~\cite{Kolb:2003dz,Song:2007ux,Dusling:2007gi,Holopainen:2010gz,Schenke:2010rr,Romatschke:2007mq,Niemi:2015qia,Jeon:2015dfa,Romatschke:2017ejr}, and hadronic cascade model~\cite{Bass:1998ca,Bleicher:1999xi,Weil:2016zrk} leads to the most successful description of a wide range of observables, e.g., particle spectrum, centrality dependence of average particle transverse momentum, and multi-particle correlations in heavy-ion collisions~\cite{ALICE:2016kpq,Acharya:2017gsw,Acharya:2017zfg,Acharya:2020taj,ALICE:2021klf,ALICE:2021adw,ALICE:2013mez,ALICE:2011ab}. By employing global Bayesian analysis, our understanding of the underlying theory has advanced such that we can infer the parameters of the multi-stage model, including those quantifying the transport properties of the QGP, using experimental measurements~\cite{Bernhard:2016tnd,Bernhard:2019bmu,Parkkila:2021tqq,Parkkila:2021yha}. Despite the studies describing both heavy AA and pA collisions in a single framework~\cite{Moreland:2018gsh}, the origin of the flow-like correlations is still under debate. It is unclear whether the flow-like behavior originates from the early stages of the collision in the context of CGC-EFT~\cite{Dusling:2012cg,Bzdak:2013zma} or whether it develops during the collective evolution period, where causal hydrodynamics is applicable~\cite{Greif:2017bnr,Mantysaari:2017cni}. Both scenarios may be responsible for the observed correlations in the final state~\cite{Greif:2017bnr}. Although collective models are successful in describing available two-particle correlation data from small-system collisions, they predict an opposite sign for four-particle cumulants compared to the experiment~\cite{Khachatryan:2016txc,ATLAS:2017rtr,Zhao:2017rgg}. On the other hand, a semi-analytical toy model based on the Gubser hydrodynamic solution~\cite{Gubser:2010ze,Gubser:2010ui} can explain the two- and four-particle correlations in pp collisions~\cite{Taghavi:2019mqz}. In particular, this model has explained the relationship between the sign of the four-particle cumulants and the fluctuation in the initial state. 

Besides the models based on the causal hydrodynamic framework in describing the collective evolution in small-system collisions, there are other attempts to explain the flow-like signals using alternative descriptions. For instance, a study based on the A Multi-Phase Transport model (AMPT)~\cite{Lin:2004en} leads to satisfactory agreement with the experiment~\cite{OrjuelaKoop:2015jss}. The applicability of fluid-dynamical simulations and partonic cascade models in small-system collisions has been explored in Ref.~\cite{Gallmeister:2018mcn}. In the context of kinetic theory with isotropization-time approximation, the model can smoothly explain the long-range correlations by fluid-like (hydrodynamic) excitations for Pb--Pb collisions and particle-like (or non-hydrodynamic) excitations for pp or p--Pb collisions~\cite{Kurkela:2019kip,Kurkela:2020wwb,Ambrus:2021fej}. Another potential description for the collectivity in small systems is provided by the PYTHIA~8 event generator, in which interacting strings repel one another transversely in an mechanism dubbed ``string shoving''~\cite{Bierlich:2017vhg,Bierlich:2019ixq}. The repulsion of the strings causes microscopic transverse pressure, giving rise to the long-range correlations. The string shoving approach in PYTHIA~8 successfully reproduces the near-side ($\Delta\varphi\sim0$) ridge yield in measurements by ALICE~\cite{ALICE:2021nir} and CMS~\cite{Khachatryan:2016txc}. A systematic mapping of correlation effects across collision systems of varying sizes is currently underway on the theoretical side (see, for example, \cite{Schenke:2020mbo}). The quantitative description of the full set of experimental data has not been achieved yet. A summary of various explanations for the observed correlations in small- system collisions is given in~\cite{Strickland:2018exs,Loizides:2016tew,Nagle:2018nvi}. 

%Another crucial evidence of a strongly interacting medium in small collision systems would be the presence of jet quenching~\cite{Gyulassy:1990ye,Wang:1991xy}. However, no evidence has been observed in either high-multiplicity pp or p--Pb collisions~\cite{Adam:2014qja,Khachatryan:2016odn,Adam:2016jfp,Adam:2016dau,Acharya:2017okq}. A study with two-particle angular correlations in a short range around $(\Delta\eta$, $\Delta\varphi)=(0,0)$ is a good tool for studying jet fragmentation~\cite{Adam:2016tsv}.

%On the experimental side, flow extraction methods in these small systems have not been fully understood due to the strong jet fragmentation bias. 
Measurements of anisotropic flow in small-system collisions are strongly affected by non-flow effects, predominantly originating from correlations among the constituents of jet fragmentation processes.
In case of two-particle correlations, the non-flow contribution is usually suppressed by requiring a large $\Delta\eta$ gap between the two particles. This separation in pseudorapidity is also widely used in cumulant methods~\cite{Bilandzic:2010jr, Acharya:2019vdf}. However, this $\Delta\eta$ gap method for both two-particle correlations and cumulants removes the non-flow contribution only on the near side and not on the away side~($\Delta\varphi\sim\pi$). Later, a low-multiplicity template fit was proposed to remove non-flow contributions on the away-side~\cite{ATLAS:2015hzw, ATLAS:2016yzd, ATLAS:2018ngv}. This method takes into account that the yield of jet fragments increases with the multiplicity~\cite{CMS:2013ycn, ALICE:2013tla, ALICE:2014mas}.
By using the template fit method, we can subtract the yield of away-side jet fragments, provided that the distribution of jet fragments has a shape that is independent of the chosen multiplicity class and can be described by the low multiplicity template. 
However, it is possible that the shape of the away-side jet fragments may be modified with respect to the low-multiplicity template for a given multiplicity selection. The effect of this shape modification, which was previously ignored, will be discussed and quantified in this paper.

%This template fit method allows one to subtract the enhanced away-side jet fragments which are estimated from low-multiplicity events, as long as the jet shape is not heavily modified between the template and in the target multiplicity. The previously ignored possible effect driven by the minor observed modification will be quantified.
%This template fit method allows one to subtract the enhanced away-side jet fragments in high-multiplicity with respect to low-multiplicity events, possibly testing a lower limit of event multiplicity on the flow signal.

As an extension of the studies of the near-side long-range ridge and jet fragmentation yields in pp collisions at $\sqrt{s}=13\,\mathrm{TeV}$, which were reported in Ref.~\cite{ALICE:2021nir}, this article studies the interplay of the jet production and collective effects, i.e., short- and long-range correlations simultaneously in high-multiplicity pp collisions at $\sqrt{s} =13$ TeV and $\sqrt{s_{\mathrm{NN}}}=5.02$ TeV p--Pb collisions. The article also reports flow coefficients extracted for collisions, which were tagged with different event-scale selection. The event-scale selection requires a minimum transverse momentum of the leading particle or the reconstructed jet at mid-rapidity, which is expected to bias the impact parameter of pp collisions to be smaller on average~\cite{Sjostrand:1986ep,Frankfurt:2003td,Frankfurt:2010ea}. At the same time, the transverse momentum of the leading particle or the reconstructed jet provides a measure of the momentum transfer ($Q^2$) in the hard parton scattering~\cite{Chatrchyan:2012tt, Chatrchyan:2011id}. The transverse momentum threshold implies a higher $Q^2$ for the collision. Such events with large $Q^2$ may, on average, have a lower impact parameter, $b$, than pp events without any requirement on $Q^2$~\cite{Frankfurt:2003td}.

This paper is organized as follows. First, the experimental setup and analysis method are described in Sec.~\ref{sec:experiment} and \ref{sec:ana}, respectively. Then, the estimation of systematic uncertainties is discussed in Sec.~\ref{sec:uncertainties}. The results and comparisons of measurements with model calculations are presented in Sec.~\ref{sec:results}. Finally, the results are summarized in~Sec.~\ref{sec:summary}.

 

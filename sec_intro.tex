% !TEX root = paper.tex

\section{Introduction}
\label{sec:intro}

Significant correlations are observed between particles emitted over a wide pseudorapidity range in high-energy nucleus--nucleus collisions at RHIC~\cite{Adams:2005dq,Adcox:2004mh,Arsene:2004fa,Back:2004je} and LHC~\cite{Abelev:2012di, Abelev:2014pua, ATLAS:2011ah}. The origin of these observations are collective effects, which are related to the formation of a strongly interacting quark-gluon plasma (QGP), which exhibits hydrodynamic behavior (see the reviews~\cite{Romatschke:2007mq,Jeon:2015dfa,Romatschke:2017ejr}). Recent theoretical~\cite{Niemi:2015qia,Bernhard:2016tnd,Bernhard:2019bmu,Parkkila:2021tqq,Parkkila:2021yha} and experimental~\cite{ALICE:2016kpq,Acharya:2017gsw,Acharya:2017zfg,Acharya:2020taj} advancements have contributed significantly to the understanding of the transport properties of the QGP.
Similar long-range correlations are also observed in high-multiplicity proton--proton (pp)~\cite{Aad:2015gqa,Khachatryan:2015lva,Khachatryan:2016txc,Acharya:2019vdf}, proton--nucleus (pA)~\cite{Abelev:2012ola,Aad:2014lta,Aaboud:2016yar,Khachatryan:2016ibd}, and light nucleus--nucleus collisions~\cite{PHENIX:2018lia,Aidala:2017ajz}. The fact that these correlations extend over a large range in pseudorapidity implies that they originate from early times in these collisions and thus suggest that hydrodynamic behavior is present even in these small systems, although the volume and lifetime of the medium produced in such a collision system are expected to be small, and there are other mechanisms which can produce similar flow-like signals~\cite{Busza:2018rrf,Nagle:2018nvi}.

These flow signals in high-multiplicity pp and p--Pb events have been attributed to initial-state or final-state effects. Initial-state effects, usually attributed to gluon saturation~\cite{Dusling:2012cg,Bzdak:2013zma}, can form long-range
correlations along the longitudinal direction. The final-state effects might be parton-induced interactions~\cite{Arbuzov:2011yr} or collective phenomena due to hydrodynamic behavior of the produced matter arising in a high-density system possibly formed in these collisions~\cite{Weller:2017tsr,Zhao:2017rgg}. 
Hybrid models implementing both effects are generally used in hydrodynamic simulations~\cite{Greif:2017bnr,Mantysaari:2017cni}. EPOS LHC describes collectivity in small systems with a parameterized hydrodynamic evolution of the high-energy density region, so called ``core'', formed by many color string fields~\cite{Pierog:2013ria}.
The proton shape and its fluctuations are also important to model small systems~\cite{Mantysaari:2017cni}.
To understand the influence of initial- or final-state effects, and to possibly disentangle the two, a quantitative description of the measurements in small systems~\cite{Schenke:2019pmk,Schenke:2020mbo} needs to account for details of the initial state.
Besides the hybrid models mentioned above, alternative approaches were developed to describe collectivity in small systems. A microscopic model for collectivity was implemented in the PYTHIA~8 event generator, which is based on interacting strings (string shoving) and is called the “string shoving model”~\cite{Bierlich:2017vhg}. In this model, strings repel each other in the transverse direction, which results in microscopic transverse pressure and, consequently, in long-range correlations. PYTHIA~8 with string shoving can qualitatively reproduce the near-side ($\Delta\varphi\sim0$) ridge yield measured by the CMS Collaboration~\cite{Khachatryan:2016txc}.
Systematic studies of these correlation effects from small to large systems are being performed in a theory side~(see i.e. \cite{Schenke:2020mbo}).
The quantitative description of the full set of experimental data has not been achieved yet. A summary of various explanations for the observed correlations in small systems is given in~\cite{Strickland:2018exs,Loizides:2016tew,Nagle:2018nvi}. 

In experiment side, flow extraction methods in these small systems have not been fully understood due to the strong jet fragmentation bias. In order to remove the remaining non-flow contributions in i.e. cumulant methods~\cite{Bilandzic:2010jr, Acharya:2019vdf}, we perform a flow-extraction method using a low-multiplicity template. Flow coefficients in various high-multiplicity classes are extracted from the low-multiplicity template fit method~\cite{ATLAS:2015hzw,ATLAS:2016yzd}. This template fit method allows one to subtract the enhanced away-side jet yields in high-multiplicity with respect to low-multiplicity events, possibly testing a lower limit of event multiplicity on the flow signal.

It is expected that final-state interactions affect also produced jets if they are the source of collectivity in small systems. Proving the presence of jet quenching~\cite{Gyulassy:1990ye,Wang:1991xy} would be another crucial evidence of the existence of a high-density strongly-interacting system, possibly a QGP, in high-multiplicity pp collisions. However, there is no evidence observed so far for the jet quenching effect in high-multiplicity pp and p--Pb collisions~\cite{Khachatryan:2016odn,Adam:2016jfp,Adam:2016dau,Acharya:2017okq}. Jet fragmentation can be studied in two-particle angular correlations in short-range correlations around $(\Delta\eta$, $\Delta\varphi)=(0,0)$~\cite{Adam:2016tsv}.  

In this report as an extension of Ref.~\cite{ALICE:2021nir}, the interplay of jet production and collective effects in small systems, long- and short-range correlations are studied simultaneously in high-multiplicity pp collisions at $\sqrt{s} =13$ TeV and $\sqrt{s_{\mathrm{NN}}}=5.02$ TeV p-Pb collisions. 
In addition, the flow and near-side jet-like correlations with the event-scale selection are reported. The event-scale selection is done by requiring a minimum transverse momentum of the leading particle or the reconstructed jet at mid-rapidity, which is expected to bias the impact parameter of pp collisions to be smaller on average~\cite{Sjostrand:1986ep,Frankfurt:2010ea}. At the same time, the transverse momentum of the leading particle or the reconstructed jet is a measure of the momentum transfer in the hard parton scattering~\cite{Chatrchyan:2012tt,Chatrchyan:2011id}.
% want to add here this is an extended work to extract flow based on the same analysis on 2D correlation functions?
The experimental setup and analysis method are described in Sec.~\ref{sec:experiment} and \ref{sec:ana}, respectively. The sources of systematic uncertainties are discussed in Sec.~\ref{sec:uncertainties}. The results and comparisons with model calculations of the measurements are presented in Sec.~\ref{sec:results}. Finally, results are summarized in~Sec.~\ref{sec:summary}.

 